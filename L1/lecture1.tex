\documentclass[11pt,aspectratio=169]{beamer}
%\documentclass[11pt,aspectratio=169,handout]{beamer}

\input{../Macros/main}
\subtitle{\vspace{2.1em}Lecture 1: Introduction}

\begin{document}

 \begin{frame}[plain]
  \titlepage
 \end{frame}
 
 \section{Course Mechanics}
  \begin{frame}{Course Mechanics}
   \begin{itemize}[<+->]
    \setlength{\itemsep}{1em}
    \item Course website
    \begin{itemize}[<.->]
     \item \href{https://ece.uwaterloo.ca/~smzahedi/crs/ece750}{\url{https://ece.uwaterloo.ca/~smzahedi/crs/ece750/}}
     \item All course information, lecture notes, assignments, etc.  
    \end{itemize}   
    \item Office hours
    \begin{itemize}[<.->]
     \item Book appointment online
     \item Or catch me after class, or send me email to setup meeting
    \end{itemize}
    \item Prerequisites 
    \begin{itemize}[<.->]
    	 \item None
     \item Helpful knowledge: algorithms, probability, comp. complexity, and optimization
    \end{itemize} 
    \item Anti-requisites 
    \begin{itemize}[<.->]
     \item ECON.412 (Topics in Game Theory), CO.456 (Introduction to Game Theory), CS.886 (Multi-agent Systems), and MSCI.724 (Game Theory and Recent App)
    \end{itemize} 
   \end{itemize}
   \only<1->{
    \begin{textblock*}{0.5in}(12cm,1.75cm) % {block width} (coords)
     \qrcode[height=0.5in]{https://ece.uwaterloo.ca/\string~smzahedi/crs/ece750}
    \end{textblock*}
   }
  \end{frame}
 
 \section{Course Outline}
  \begin{frame}{Tentative Topics}
   \begin{itemize}[<+->]
    \setlength{\itemsep}{1em}
    \item\alert<.>{Introduction to multi-agent systems}
    \begin{itemize}[<.->]
     \item Overview of game theory and mechanism design, rationality and self-interest, utility theorem, risk attitudes
    \end{itemize}
    \item\alert<.>{Games in normal form}
    \begin{itemize}[<.->]
     \item Pure and mixed-strategy Nash equilibrium, iterative elimination of dominated strategies, price of anarchy, correlated equilibrium, computing solution concepts of normal-form games
    \end{itemize}
    \item\alert<.>{Games in extensive form}
    \begin{itemize}[<.->]
     \item Perfect and imperfect-information games, finite and infinite-horizon games, subgame-perfect equilibrium, backward induction, one-shot deviation principle
    \end{itemize}
   \end{itemize}
  \end{frame}
 
  \begin{frame}{Tentative Topics (cont.)}
   \begin{itemize}[<+->]
    \setlength{\itemsep}{1em}
    \item\alert<.>{Beyond normal and extensive-form games}
    \begin{itemize}[<.->]
     \item Repeated games, stochastic games, Bayesian games, congestion games, trigger strategies, folk theorems, Bayes-Nash equilibrium, auctions, optimal auctions, revenue-equivalence theorem, incentive compatibility, VCG mechanisms
    \end{itemize}
    \item\alert<.>{Learning in multi-agent systems}
    \begin{itemize}[<.->]
     \item Multi-agent reinforcement learning, fictitious play, Bayesian learning, regret-minimization learning
    \end{itemize}
   \end{itemize}
  \end{frame}
 
  \begin{frame}{Textbook and References}
   \footnotesize
   \begin{itemize}
    \setlength{\itemsep}{1em}
    \item Y. Shoham and K. Leyton-Brown, \textbf{Multi-agent Systems: Algorithmic, Game-theoretic, and Logical Foundations} \href{http://www.masfoundations.org/mas.pdf}{(available online)}
    \item	N. Nisan, et al. \textbf{Algorithmic Game Theory} \href{https://www.cs.cmu.edu/~sandholm/cs15-892F13/algorithmic-game-theory.pdf}{(available online)}
    \item T. Roughgarden, \textbf{Twenty Lectures on Algorithmic Game Theory} \href{https://timroughgarden.org/notes.html}{(notes available online)}
    \item D. Fudenberg and D. Levine, \textbf{The Theory of Learning in Games}
    \item D. Fudenberg and J. Tirole, \textbf{Game Theory}
    \item M. J. Osborne and A. Rubinstein, \textbf{A Course in Game Theory} \href{https://books.osborne.economics.utoronto.ca/}{(available online)}
   \end{itemize}
  \end{frame}
 
  \begin{frame}{Course Requirements}
   \begin{itemize}
    \item<+-> Graduate offering (available to undergraduate students)
    \begin{itemize}[<+-| alert@+>]
     \item Quizzes 5\%
     \item Assignments 20\%
     \item Research project 45\%
     \item Final exam 30\%
    \end{itemize}
    \vspace{1.4em}
    \item<+-> Undergraduate offering (available to MEng students)
    \begin{itemize}[<+-| alert@+>]
     \item Quizzes 5\%
     \item Assignments 20\%
     \item Survey project 25\%
     \item Final exam 50\%
    \end{itemize}
   \end{itemize}
  \end{frame}
 
  \begin{frame}{Survey Project (Undergraduate Offering)}
   \begin{itemize}
   \setlength{\itemsep}{1em}
    \item Goal is to review existing literature in sub-area of multi-agent systems and possibly explore open research questions in that sub-area
    \item Two milestones: proposal (15\%) and final written report (85\%)
   \end{itemize}
  \end{frame}
 
  \begin{frame}{Research Project (Graduate Offering)}
   \begin{itemize}
   \setlength{\itemsep}{1em}
   \item Goal is to try to do something novel, rather than merely surveying existing work
   \item Only real constraint is that it has to have something to do with covered material
   \item Projects may be theoretical, experimental (based on simulations), experimental (based on real-world data), a useful software artifact, or any combination thereof
   \item Three milestones: proposal (15\%), oral progress report (optional), and final written report (85\%)
   \item Creativity is encouraged
   \end{itemize}
  \end{frame}
 
 \section{Overview of Game Theory and Mechanism Design}
 
  \begin{frame}{Questions and Problems in (Computational) Game Theory}
   \begin{itemizes}
    \item<+-> How should we represent games (i.e., strategic settings)?
    \begin{itemize}
     \item<.-> Game-theoretic representations not always concise enough
    \end{itemize} 
	\item<+-> What does it mean to solve a game?
	\begin{itemize}
	 \item<.-> Solution concepts from game theory, e.g., Nash equilibrium
	\end{itemize}
	\item<+-> How computationally hard is it to solve a games?
	\begin{itemize}
	 \item<.-> Can we solve them approximately?
	\end{itemize}
	\item<+-> Is there a role for (machine) learning in games?
    \item<+-> What types of modeling problems do we face when addressing real-world games?
   \end{itemizes}
  \end{frame}
  
  %TODO ML/AI/GT
  
  
  \begin{frame}{Optimization vs. Game Theory}
   \begin{itemize}[<+->]
    \item {\color{blue}Optimization theory}: Optimize a single objective over a decision variable
    \begin{equation*}
     \begin{aligned}
      &\min.			& 	&u(x)\\
      &\text{s.t.}		&	&x \in X
     \end{aligned}
    \end{equation*}
    \item {\color{blue}Game theory}: Study of multi-agent decision problems
    \begin{itemizes}[0.35em]
     \item Model \alert{cooperation} and \alert{competition} between \alert{intelligent} and \alert{rational} decision makers
     \item $n$ agents, each chooses some $x_i \in X_i$, and has a utility function
     \begin{equation*}
      u_i(x_i,x_{-i}), ~~~ x_{-i} = (x_1,\dots,x_{i-1},x_{i+1},\dots,x_n)
     \end{equation*}
     \item What are the possible outcomes?
     \item Steady-state, stable operating point, characteristics?
     \item How do you get there (learning dynamics, computation of equilibrium)?
    \end{itemizes}
   \end{itemize}    
  \end{frame}
  
  \begin{frame}{Guess 2/3 of the Average (2/3-Beauty Contest Game)}
   \begin{itemize}
    \item Everyone writes down an integer between 1 and 100
    \item Person closest to 2/3 of the average wins
    \pause
    \item Example:
    \begin{itemize}
     \item A says 50 - B says 10 - C says 90
     \item 2/3 of average(50, 10, 90) = 2/3 $\times$ 50 = 33.33
     \item A is the closest ($\vert 50-33.33\vert$ = 16.67), so A wins
    \end{itemize}
   \end{itemize}
  \end{frame}
  
  \begin{frame}{Mechanism Design}
   \begin{itemize}[<+->]
   \setlength{\itemsep}{1em}
    \item {\color{blue} Inverse Game Theory}: Design of game forms to implement desirable outcomes
    \begin{itemize}[<.->]
     \item E.g.\ to incentivize independent agents to reveal their types truthfully
    \end{itemize}
    \item In Economics, MD is all about designing the right incentives
    \begin{itemize}[<.->]
     \item Mechanisms map \alert{signals} from independent agents into allocations and payments
     \item Optimal Mechanisms (Myerson): Design a mechanism that maximizes profits
     \item Efficient Mechanisms (Vickrey-Clarke-Groves (VCG) Mechanisms): Design a mechanism to maximize a \alert{social} or system-wide objective
    \end{itemize}
    \item In CS/Engineering, focus is more on the design of efficient decentralized protocols that take into account incentives
    \begin{itemize}[<.->]
     \item Mechanisms in networks, distributed and online mechanisms, mechanisms that operate with limited information
    \end{itemize}
   \end{itemize}
  \end{frame}
  
  \begin{frame}{Example: Single-item Auction}
   \begin{itemize}
   \setlength{\itemsep}{1.5em}
    \item \alert{Sealed-bid auction}: every bidder submits bid in sealed envelope
    \item \alert{First-price} sealed-bid auction: highest bid wins, pays amount of own bid
    \item \alert{Second-price} sealed-bid auction: highest bid wins, pays amount of 2nd-highest bid
   \end{itemize}
  \end{frame}
  
  \begin{frame}{Which Auction Generates More Revenue?}
   \begin{itemize}
   \setlength{\itemsep}{1.5em}
    \item Each bid depends on
    \begin{itemize}
     \item bidder's true valuation for the item (utility = valuation - payment)
     \item bidder's beliefs over what others will bid
     \item and $\dots$ auction mechanism used
    \end{itemize}
    \item In first-price auction, it does not make sense to bid your true valuation
    \begin{itemize}
     \item Even if you win, your utility will be 0
    \end{itemize}
    \item In second-price auction, it always makes sense to bid your true valuation \\ (we will see this later)
   \end{itemize}
  \end{frame}
  
  \begin{frame}{What is Mechanism Design Again?}
   \begin{itemize}[<+->]
   \setlength{\itemsep}{1.2em}
    \item Designing mechanism $=$ designing a game
    \item We can use game theory to predict what will happen under mechanism
    \begin{itemize}
     \item If agents act strategically
    \end{itemize}     
    \item When is a mechanism ``good''?
    \begin{itemize}
     \item Should it result in outcomes that are ``good'' for \alert{reported} or for \alert{true} preferences?
     \item Should agents ever end up lying about their preferences (in game-theoretic solution)?
     \item Should it always generate the best allocation?
     \item Should agents ever burn money?(!?)
    \end{itemize}
    \item Can we solve for the optimal mechanism?
   \end{itemize}
  \end{frame}
  
  \begin{frame}{Acknowledgment}
   \begin{itemize}
    \setlength{\itemsep}{1em}
    \item This lecture is a slightly modified version of ones prepared by
    \begin{itemizes}
     \item Asu Ozdaglar \href{https://ocw.mit.edu/courses/electrical-engineering-and-computer-science/6-254-game-theory-with-engineering-applications-spring-2010/index.htm}{[MIT 6.254]}
     \item Vincent Conitzer \href{https://courses.cs.duke.edu/spring16/compsci590.4/}{[Duke CPS 590.4]}
    \end{itemizes}
   \end{itemize}
  \end{frame}
 
\end{document}
